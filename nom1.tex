\documentclass[10pt,a4paper,oneside]{book}
\usepackage{custom}

\makeatletter
\newenvironment{sqcases}{%
  \matrix@check\sqcases\env@sqcases
}{%
  \endarray\right.%
}
\def\env@sqcases{%
  \let\@ifnextchar\new@ifnextchar
  \left\lbrack
  \def\arraystretch{1.2}%
  \array{@{}l@{\quad}l@{}}%
}
\makeatother


\begin{document}
\title{Эхний ном}
\author{Зохиогчийн нэр}
\date{2016 он}
\maketitle
\chapter{Оршил бодлогууд}

\Problem Бодит $x$ тооны хувьд $\sec x - \tg x = 2$ бол $\sec x + \tg x$-г ол.

\TheSolution
$(\sec x + \tg x)(\sec x - \tg x)$ үржвэрийг авч үзье.
\begin{equation*}
\begin{split}
(\sec x + \tg x)(\sec x - \tg x)= \sec^2 x - \tg^2 x =\\
= \frac{1}{\cos^2 x} - \frac{\sin^2 x}{\cos^2 x}
= \frac{1-\sin^2 x}{\cos^2 x} = \frac{\cos^2 x}{\cos^2 x} = 1
\end{split}
\end{equation*}
Өөрөөр хэлбэл $\forall x \in D(f)$ хувьд $\sec^2 x - \tg^2 x = 1$ болно. Иймд 
\begin{equation*}
\sec x + \tg x = \frac{1}{\sec x - \tg x} = \frac{1}{2}
\end{equation*}
 болно.
 
\Problem $0^{\circ}<\theta< 45^{\circ}$ бол
\begin{equation*}
\begin{split}
t_1 = (\tg\theta)^{\tg\theta},\hspace{1cm} t_2 = (\tg\theta)^{\ctg\theta},\\
t_3 = (\ctg\theta)^{\tg\theta},\hspace{1cm} t_4 = (\ctg\theta)^{\ctg\theta}
\end{split}
\end{equation*}
тоонуудыг буурах эрэмбээр эрэмбэл.

\TheSolution
$\forall a > 1$ тооны хувьд $y=a^x$  функц өсөх функц ба $0 < a < 1$ үед уг функц буурах функц юм. $0^{\circ} < \theta < 45^{\circ}$ завсарт $\ctg\theta > 1 > \tg\theta > 0$ учир  $t_4 > t_3$, $t_1 > t_2$ ба $t_3 > 1 > t_1$ болно. Өөрөөр хэлбэл $t_4 > t_3 > t_1 > t_2$ болно.

\Problem Тооцоол.
\begin{enumerate}[(a)]
\item $\sin\frac{\pi}{12}, \cos\frac{\pi}{12}, \tg\frac{\pi}{12}$
\item $\cos^4\frac{\pi}{24}-\sin^4\frac{\pi}{24}$
\item $\cos 36^\circ - \cos 72^\circ$
\item $\sin 10^\circ \sin 50^\circ\sin 70^\circ$
\end{enumerate}

\TheSolution
\begin{enumerate}[(a)]
\item Давхар өнцөг болон нийлбэр ялгаварын томъёог ашиглавал
\begin{equation*}
\begin{split}
\sin\frac{\pi}{12} = \sin\left(\frac{\pi}{3}-\frac{\pi}{4}\right)=\sin\frac{\pi}{3}\cos\frac{\pi}{4}-\sin\frac{\pi}{4}\cos\frac{\pi}{3}=\\
=\frac{\sqrt{3}}{2}\cdot\frac{\sqrt{2}}{2}-\frac{1}{2}\cdot\frac{\sqrt{2}}{2}=\frac{\sqrt{6}-\sqrt{2}}{4}
\end{split}
\end{equation*}
\begin{equation*}
\begin{split}
\cos\frac{\pi}{12} = \cos\left(\frac{\pi}{3}-\frac{\pi}{4}\right) = \cos\frac{\pi}{3}\cos\frac{\pi}{4}+\sin\frac{\pi}{3}\sin\frac{\pi}{4}=\\
=\frac{1}{2}\cdot\frac{\sqrt{2}}{2}+\frac{\sqrt{3}}{2}\cdot\frac{\sqrt{2}}{2}=\frac{\sqrt{2}+\sqrt{6}}{4}
\end{split}
\end{equation*}
\begin{equation*}
\begin{split}
\tg\frac{\pi}{12}=\frac{\sin\frac{\pi}{12}}{\cos\frac{\pi}{12}}=\frac{\frac{\sqrt{6}-\sqrt{2}}{4}}{\frac{\sqrt{6}+\sqrt{2}}{4}}=\frac{\sqrt{6}-\sqrt{2}}{\sqrt{6}+\sqrt{2}}=2-\sqrt{3}
\end{split}
\end{equation*}
\item
\begin{equation*}
\begin{split}
\cos^4\frac{\pi}{24}-\sin^4\frac{\pi}{24}=\left(\cos^2\frac{\pi}{24}+\sin^2\frac{\pi}{24}\right)\left(\cos^2\frac{\pi}{24}-\sin^2\frac{\pi}{24}\right) = 1\cdot\cos\frac{\pi}{12} = \frac{\sqrt{2}+\sqrt{6}}{4}
\end{split}
\end{equation*}
\item
\begin{equation*}
\cos 36^\circ - \cos 72^\circ = \frac{2\left(\cos 36^\circ - \cos 72^\circ\right)\left(\cos 36^\circ + \cos 72^\circ\right)}{2\left(\cos 36^\circ + \cos 72^\circ\right)}=\frac{2\cos^2 36^\circ - 2\cos^2 72^\circ}{2\left(\cos 36^\circ + \cos 72^\circ\right)}
\end{equation*}
Давхар өнцгийн томьёо ашиглавал
\begin{equation*}
\frac{2\cos^2 36^\circ - 2\cos^2 72^\circ}{2\left(\cos 36^\circ + \cos 72^\circ\right)} = \frac{\cos 72^\circ + 1 - \cos 144^\circ -1}{2\left(\cos 36^\circ + \cos 72^\circ\right)} = \frac{\cos 72^\circ + \cos 36^\circ}{2\left(\cos 36^\circ + \cos 72^\circ\right)} = \frac{1}{2}
\end{equation*}
Дээрх тэнцэтгэлийг геометрийн аргаар баталж болно. Үүний тулд оройн өнцөг $\measuredangle A = 36^\circ$ ба талууд нь $AB=AC$, $BC=1$ байх адил хажуут гурвалжныг авч үзье. Уг гурвалжны $\angle B$ өнцгийн биссектрис $AC$ талтай огтлолцох огтлолцлыг $D$ гэвэл $BC=BD=AD=1, AB=2\cos 36^\circ$ ба $CD=2\cos 72^\circ$ болохыг та бүхэн бие даан батлаарай. Үр дүн нь дээрх тэнцэтгэлийн геометр баталгаа юм.
\item
\begin{equation*}
\begin{split}
8\sin 20^\circ\sin 10^\circ\sin50^\circ\sin70^\circ = 8\sin 20^\circ\cos 20^\circ\cos40^\circ\cos80^\circ =\\
= 4\sin 40^\circ\cos 40^\circ\cos 80^\circ=2\sin 80^\circ\cos 80^\circ = \sin 160^\circ = \sin 20^\circ
\end{split}
\end{equation*}
Эндээс
\begin{equation*}
\sin 10^\circ\sin 50^\circ\sin 70^\circ = \frac{1}{8}
\end{equation*}
болно.
\end{enumerate}

\Problem Илэрхийллийг хялбарчил.
\begin{equation*}
\sqrt{\sin^4 x + 4\cos^2 x}-\sqrt{\cos^4 x + 4\sin^2 x}
\end{equation*}

\TheSolution
\begin{equation*}
\begin{split}
\sqrt{\sin^4 x + 4\cos^2 x}-\sqrt{\cos^4 x + 4\sin^2 x} = \sqrt{\sin^4 x + 4\left(1-\sin^2 x\right)}-\sqrt{\cos^4 x + 4\left(1-\cos^2 x\right)}=\\
 =\sqrt{\left(2-\sin^2 x\right)^2}-\sqrt{\left(2-\cos^2 x\right)^2} = \left(2-\sin^2 x\right) - \left(2-\cos^2 x\right) = \cos^2 x - \sin^2 x = \cos 2x
\end{split}
\end{equation*}

\Problem Батал.
\begin{equation*}
1-\ctg 23^\circ = \frac{2}{1-\ctg 22^\circ}
\end{equation*}

\ASolution
\begin{equation*}
\left(1-\ctg 23^\circ\right)\left(1-\ctg 22^\circ\right) = 2
\end{equation*}
болохыг баталья.
\begin{equation*}
\begin{split}
\left(1-\ctg 23^\circ\right)\left(1-\ctg 22^\circ\right) = \left(1-\frac{\cos 23^\circ}{\sin 23^\circ}\right)\left(1-\frac{\cos 22^\circ}{\sin 22^\circ}\right)
= \frac{\sin 23^\circ - \cos 23^\circ}{\sin 23^\circ}\cdot\frac{\sin 22^\circ - \cos 22^\circ}{\sin 22^\circ} = \\
= \frac{\sqrt{2}\sin \left(23^\circ - 45^\circ\right)\sqrt{2}\sin \left(22^\circ - 45^\circ\right)}{\sin 23^\circ \cdot \sin 22^\circ}=\frac{2\sin \left(-22^\circ\right)\sin \left(-23^\circ\right)}{\sin 23^\circ \sin 22^\circ}=\frac{2\sin 22^\circ\sin 23^\circ}{\sin 23^\circ \sin 22^\circ} = 2
\end{split}
\end{equation*}

\ASolution Котангенсийн өнцгүүдийн нийлбэрийн томьёогоор
\begin{equation*}
\frac{\ctg 22^\circ\ctg 23^\circ-1}{\ctg 22^\circ+\ctg 23^\circ}=\ctg\left(22^\circ+23^\circ\right)=\ctg 45^\circ = 1
\end{equation*}
Эндээс $\ctg 22^\circ\ctg 23^\circ - 1 = \ctg 22^\circ + \ctg 23^\circ$ буюу
\begin{equation*}
1-\ctg 22^\circ - \ctg 23^\circ + \ctg 22^\circ\ctg 23^\circ = 2
\end{equation*}
болно. Өөрөөр хэлбэл
\begin{equation*}
\left(1-\ctg 23^\circ\right)\left(1-\ctg 22^\circ\right) = 2
\end{equation*}
болно.

\Problem 
\begin{equation*}
\frac{\sqrt{3}-1}{\sin x}+\frac{\sqrt{3}+1}{\cos x} = 4\sqrt{2}
\end{equation*}
тэгшитгэлийн $\left(0, \frac{\pi}{2}\right)$ завсар дахь бүх шийдийг ол.

\TheSolution
Бодлого $3(a)$ дээр бид $\cos \frac{\pi}{12}=\frac{\sqrt{2}+\sqrt{6}}{4}$ ба $\sin \frac{\pi}{12}=\frac{\sqrt{6}-\sqrt{2}}{4}$ болохыг харсан. Эдгээрийг тэгшитгэлд орлуулвал
\begin{equation*}
\frac{\frac{\sqrt{3}-1}{4}}{\sin x}+\frac{\frac{\sqrt{3}+1}{4}}{\cos x} = \sqrt{2}
\Rightarrow
\frac{\sin \frac{\pi}{12}}{\sin x} + \frac{\cos \frac{\pi}{12}}{\cos x} = 2
\Rightarrow
\end{equation*}
\begin{equation*}
\Rightarrow
\sin\frac{\pi}{12}\cos x + \cos\frac{\pi}{12}\sin x = 2\sin x\cos x
\Rightarrow
\end{equation*}
\[
\Rightarrow
\sin\left(\frac{\pi}{12}+x\right) = \sin 2x
\Rightarrow
\begin{sqcases}
\frac{\pi}{12}+x = 2x\\
\frac{\pi}{12}+x = \pi - 2x
\end{sqcases}
\Rightarrow
\begin{sqcases}
x=\frac{\pi}{12}\\
x = \frac{11\pi}{36}
\end{sqcases}
\]

\Problem
$x^2 + y^2 \leq 100$ ба $\sin (x+y) \geq 0$ нөхцлүүдийг хангах $(x, y)$ цэгүүдээс бүрдэх $\mathcal{R}$ дүрсийн талбайг ол.

\TheSolution
$x^2 + y^2 \leq 100$ тэнцэтгэл бишээр илэрхийлэгдэх дугуйг $\mathcal{C}$ гэе. $\sin (x+y) = 0$ байх зайлшгүй бөгөөд хүрэлцээтэй нөхцөл нь $x+ y = k\pi$ ($k$ нь бүхэл тоо) учир $\mathcal{C}$ дугуй $x + y = k\pi$ тэгшитгэлийг хангах параллел шулуунуудаар огтологдох ба шулуунуудын хооронд $\sin (x+y) > 0$ эсвэл $\sin(x+y) < 0$ тэнцэтгэл бишүүдийг хангах $(x, y)$ цэгүүдийн мужууд оршино. $\sin(-x-y) = -\sin (x+y)$ учир уг хоёр тэнцэтгэл бишүүдийг хангах цэгүүдийн мужууд нь координатын эхийн хувьд тэгш хэмтэй юм. Иймд бидний олох $\mathcal{R}$ дүрсийн талбай нь $\mathcal{C}$ дугуйн талбайн хагас буюу $50\pi$ болно.

\Problem
$\vartriangle ABC$-ны хувьд $\sin \frac{A}{2}\leq\frac{a}{b+c}$ болохыг харуул.

\TheSolution
Гурвалжны хувьд синусын өргөтгөсөн теорем ёсоор
\begin{equation*}
\frac{a}{b+c}=\frac{\sin A}{\sin B + \sin C}
\end{equation*}
Синусуудын нийлбэрийн томьёо болон давхар өнцгийн томьёог ашиглавал
\begin{equation*}
\frac{a}{b+c} = \frac{2\sin \frac{A}{2}\cos \frac{A}{2}}{2\sin \frac{B+C}{2}\cos\frac{B-C}{2}}=\frac{\sin\frac{A}{2}}{\cos\frac{B-C}{2}}
\end{equation*}
Энд $0\leq |B-C|<180^\circ \Rightarrow 0<\cos\frac{B-C}{2}\leq 1$ тул
\begin{equation*}
\frac{\sin\frac{A}{2}}{\cos\frac{B-C}{2}}\geq\sin\frac{A}{2} \Rightarrow \frac{a}{b+c} \geq \sin\frac{A}{2}
\end{equation*}
болно. Үүнтэй адилаар
\begin{equation*}
\sin\frac{B}{2} \leq \frac{b}{c+a}\hspace{8mm} \text{ба}\hspace{8mm} \sin\frac{C}{2}\leq \frac{c}{a+b}
\end{equation*}
болно.

\Problem
$\left[-\frac{\pi}{4},\frac{\pi}{4}\right]$ завсрыг $I$ гэе. $\left[-1,1\right]$ завсарт тодорхойлогдсон $f\left(\sin 2x\right)=\sin x + \cos x$ чанарыг хангах $f$ функцийг ол. $I$ завсарт $f\left(\tan^2 x\right)$ функцийг хялбарчил.

\TheSolution
\begin{equation*}
\left[f\left(\sin 2x\right)\right]^2 = \left(\sin x + \cos x\right)^2 = \sin^2 x + \cos^2 x + 2\sin x\cos x = 1+ \sin 2x
\end{equation*}
$x \in I$ байх үед $\sin 2x \in [-1, 1]$ байна. $\sin 2x = t$ гэвэл $t \in [-1, 1]$ ба $[f(t)]^2 = 1+t$ болно. Эндээс $f(t) = \sqrt{1+t}$ болно.

$-\frac{\pi}{4}\leq x\geq\frac{\pi}{4}$ үед $-1\leq\tg x\geq1 x$ байх ба $0\geq \tg^2 x \geq 1$ тул
\begin{equation*}
f(\tg^2 x) = \sqrt{1+\tg^2 x}=\sqrt{1+\frac{\sin^2 x}{\cos^2 x}}=\sqrt{\frac{\cos^2 x + \sin^2 x}{\cos^2 x}} = \frac{1}{\cos x}= \sec x
\end{equation*}
болно.

\Problem
$\forall x \in \mathbb{R}; \forall k \in \mathbb{N}$ үед
\begin{equation*}
f_k(x) = \frac{1}{k}\left(\sin^k x + \cos^k x\right)
\end{equation*}
бол
\begin{equation*}
f_4(x)-f_6(x) = \frac{1}{12}
\end{equation*}
болохыг батал.

\TheSolution
Бид уг тэнцэтгэлийг $12f_4(x) - 12f_6(x)=1$ гэж баталья.
\begin{equation*}
\begin{split}
12\cdot\frac{1}{4}\left(\sin^4 x+\cos^4 x\right)-12\cdot\frac{1}{6}\left(\sin^6 x + \cos^6 x \right)=3\left(\sin^4 x + \cos^4 x\right) - 2\left(\sin^6 x + \cos^6 x\right) =\\
= 3\left[\left(\sin^2 x + \cos^2 x\right)^2 - 2\sin^2 x \cos^2 x\right] - 2\left(\sin^2 x + \cos^2 x\right)\left(\sin^4 x - \sin^2 x\cos^2 x + \cos^4 x\right) =\\
= 3-6\sin^2 x\cos^2 x - 2\left[\left(\sin^2 x + \cos^2 x\right)^2 - 3\sin^2 x\cos^2 x\right] = 3 - 2 = 1
\end{split}
\end{equation*}

\Problem
[AIME 2004, Jonathan Kane] $15\times 36$ хэмжээтэй $ABCD$ тэгш өнцөгтөд нэгж радиустай тойрог агуулагдана. Тэгвэл уг тойрог $AC$ диагональтай огтлолцохгүй байх магадлалыг ол.

\Note
Тойрог бүхлээрээ тэгш өнцөгтөд агуулагдаж байхын тулд тойргийн төв $13\times 34$ харьцаатай тэгш өнцөг дотор оршино. Иймд бидний олох ёстой магадлал тойргийн төвөөс $AC$ диагональ хүртэлх зай $1$-ээс их байх магадлалгыг олохтой адил юм. Энэхүү магадлал нь $13\times 34$ тэгш өнцөгтийн сонгогдсон нэг цэгээс $\triangle ABC$, $\triangle CDA$ -ны тал бүр хүртэлх зай $1$-ээс их байх магадлалтай тэнцүү юм. $|AB| = 36; |BC| = 15 \Rightarrow |AC| = 39$ болно. $\triangle ABC$ -ны дотор талд талуудаас нэгж зайтай байх шулуунуудыг татаж огтлолцлын цэгийг $E, F, G$ гэе. Тэгвэл $\triangle ABC$, $\triangle EFG$ гурвалжнууд төсөөтэй гурвалжнууд байна. Иймд бидний олох магадлал
\begin{equation*}
\frac{2\cdot S_{\triangle EFG}}{13 \cdot 34} = \left(\frac{|EF|}{|AB|}\right)^2\cdot \frac{2\cdot S_{\triangle ABC}}{13\cdot 34} = 
\left(\frac{|EF|}{|AB|}\right)^2\cdot \frac{270}{221}
\end{equation*}
болно. $E, F, G$ цэгүүд нь $\triangle ABC$-ны биссектрисүүд дээр орших ба биссектрисүүдийн огтлолцлын цэг буюу уг гурвалжинд багтсан тойргийн төвийг $I$ гэж тэмдэглэе.

\ASolution
$E, F$ цэгүүдийн $AB$ талд дээрх проекцийг нь харгалзан $E_1, F_1$ гэе. Тэгвэл $|BF_1| = |FF_1| = |EE_1| = 1$ болно. $\angle EAB = \theta$ гэвэл $\angle CAB = 2\theta$, $\sin 2\theta = \frac{5}{12}$, $\cos 2\theta = \frac{12}{13}$ болно.  Иймд
\begin{equation*}
\tg \theta = \frac{1-\cos 2\theta}{\sin 2\theta} =\frac{1}{5}
\end{equation*}
болох ба $\frac{|EE_1|}{AE_1} = \tg \theta = \frac{1}{5} \Rightarrow |AE_1| = 5$ болно. Иймд $|EF| = |E_1 F_1| = 30$ гэдгээс бидний олох магадлал $\frac{375}{442}$ болно.

\ASolution
$A = (0,0), B=(36,0), C = (36,15)$ гэе. $E$ цэг $\angle CAB$ өнцгийн биссектрис дээр орших учир
\begin{equation*}
|\overrightarrow{AB}|\overrightarrow{AC} + |\overrightarrow{AC}|\overrightarrow{AB} = 36\cdot [36,15]+ 39\cdot [36, 0] = 36\cdot 15 \cdot [5, 1]
\end{equation*}
вектортой параллел байна. Иймд $AE$ хэрчмийн налуу нь $\frac{1}{5}$ болох ба $|EE_1| = 5$ болж өмнөх бодолттой адилаар магадлал нь $\frac{375}{442}$ болно.

\ASolution
$E, F, G$ цэгүүд нь $\triangle ABC$ -ны биссектрисүүд дээр орших тул $I$ цэг нь $\triangle EFG$ -нд багтсан тойргийн төв болно. Иймд хэрэв $\triangle ABC$ -нд багтсан тойргийн радиусыг $r$ гэвэл $\triangle EFG$ -нд багтсан тойргийн радиус $r-1$ болно. Төсөөтэй гурвалжнуудын чанар ёсоор бидний олох магадлал $\left(\frac{r-1}{r}\right)^2\cdot \frac{270}{221}$ болно.
\begin{equation*}
r(|AB|+|BC|+|CA|) = 2(S_{\triangle AIB} + S_{\triangle BIC} + S_{\triangle CIA}) = 2S_{\triangle ABC} = |AB|\cdot|BC|
\end{equation*}
учир $r = 6$ болно. Иймд бидний олох магадлал $\frac{375}{442}$ болно.

\Problem
[AMC12, 1999] $ABC$ гурвалжны хувьд
\begin{equation*}
3\sin A + 4\cos B = 6 \hspace{1cm} \text{ба} \hspace{1cm} 4\sin B + 3\cos A = 1
\end{equation*}
бол $C$ өнцгийн хэмжээг ол.

\TheSolution
Өгөгдсөн хоёр тэгшитгэлийн квадратуудын нийлбэрийг олбол
\begin{equation*}
24\left(\sin A\cos B + \cos A \sin B\right) = 12 \Rightarrow \sin(A+B) = \frac{1}{2}
\end{equation*}
\begin{equation*}
\measuredangle C = 180^\circ - \measuredangle A - \measuredangle B \Rightarrow  \sin C = \sin(A+B) \Rightarrow 
\begin{sqcases}
\measuredangle C = 30^\circ\\
\measuredangle C = 150^\circ
\end{sqcases}
\end{equation*}
болно.
Гэвч $\measuredangle C = 150^\circ$ үед $\measuredangle A < 30^\circ \Rightarrow 3\sin A + 4\cos B < \frac{3}{2}+4 < 6$  болж зөрчилдөнө. Иймд бодлогын хариу $\measuredangle C = 30^\circ$ болно.

\Problem
$\forall a \neq \frac{k\pi}{2}, k \in \mathbb{Z}$ хувьд
\begin{equation*}
\tg 3a - \tg 2a - \tg a = \tg 3a \tg 2a \tg a
\end{equation*}
болохыг батал.

\TheSolution
Дээрх тэнцэтгэл дараах тэнцэтгэлтэй эквивалент юм.
\begin{align*}
\tg 3a(1-\tg 2a \tg a) &= \tg 2a + \tg a\\
\tg 3a &= \frac{\tg 2a + \tg a}{1-\tg 2a\tg a}\\
\tg 3a &= \tg(2a + a)
\end{align*}

\Note
Ерөнхий тохиолдолд $a_1, a_2, a_3 \neq \frac{k\pi}{2}, k \in \mathbb{Z}$ тоонуудын хувьд $a_1+a_2+a_3=0$ бол $\tg a_1 + \tg a_2 + \th a_3 = \tg a_1\tg a_2\tg a_3$ тэнцэтгэл биелнэ. Үүний баталгаа 13 ба 20 дугаар бодлоготой төстэйгөөр батлагдах тул дасгал болгон бие даан хийж гүйцэтгээрэй.

\Problem
$a, b, c, d \in [0, \pi]$ тоонууд
\begin{align*}
\sin a + 7\sin b &= 4\left(\sin c + 2\sin d\right)\\
\cos a + 7\cos b &= 4\left(\cos c + 2\cos d\right)
\end{align*}
тэнцэтгэлүүдийг хангадаг бол
\begin{equation*}
2\cos(a-d) = 7\cos(b-c)
\end{equation*}
болохыг батал.

\TheSolution
Өгөгдсөн тэнцэтгэлүүдийг дараах хэлбэрт бичье.
\begin{align*}
\sin a - 8\sin d &=4\sin c - 7\sin b\\
\cos a + 7\cos b &=4\cos c - 7\cos b
\end{align*}
Эдгээрийн квадратуудын нийлбэрийг эмхтгэвэл
\begin{align*}
1+64-16\left(\cos a\cos b + \sin a\sin b\right) &= 16 + 49 - 56\left(\cos b\cos c + \sin b\sin c\right)\\
2\cos(a-d) &= 7\cos(b-c)
\end{align*}

\Problem
\begin{equation*}
\sin(x-y)+\sin(y-z)+\sin(z-x)
\end{equation*}
илэрхийллийг нэг гишүүнтээр илэрхийл.

\TheSolution
Синусуудын нийлбэрийн томьёогоор
\begin{equation*}
\sin(x-y)+\sin(y-z) = 2\sin\frac{x-z}{2}\cos\frac{x+z-2y}{2}
\end{equation*}
Давхар өнцгийн томьёогоор
\begin{equation*}
\sin(z-x) = 2\sin\frac{z-x}{2}\cos\frac{z-x}{2}
\end{equation*}
болно. Иймд
\begin{multline*}
\sin(x-y)+\sin(y-z)+\sin(z-x)
= 2\sin\frac{x-z}{2}\left[\cos\frac{x+z-2y}{2}-\cos\frac{z-x}{2}\right] = \\
= -4\sin\frac{x-z}{2}\sin\frac{z-y}{2}\sin\frac{x-y}{2}
= -4\sin\frac{x-y}{2}\sin\frac{y-z}{2}\sin\frac{z-x}{2}
\end{multline*}

\Note
Ерөнхий тохиолдолд $a+b+c=0$ байх $a, b, c \in \mathbb{R}$ тоонуудын хувьд
\begin{equation*}
\sin a + \sin b + \sin c = -4\sin\frac{a}{2}\sin\frac{b}{2}\sin\frac{c}{2}
\end{equation*}
тэнцэтгэл биелнэ. Энэ бодлого нь $a=x-y;\hspace{0.5cm} b=y-z;\hspace{0.5cm} c=z-x$ байх тухайн тохиолдол юм.

\Problem
Батал.
\begin{equation*}
(4\cos^2 9^\circ-3)(4\cos^2 27^\circ - 3) = \tg 9^\circ
\end{equation*}

\TheSolution
\begin{equation*}
\cos 3x = 4\cos^3 x - 3\cos x \Rightarrow 4\cos^2 x - 3 = \frac{\cos 3x}{\cos x}, \forall x \neq(2k+1)\cdot 90^\circ
\end{equation*}
болно. Иймд
\begin{equation*}
(4\cos^2 9^\circ - 3)(4\cos^2 27^\circ - 3) = \frac{\cos 27^\circ}{\cos 9^\circ}\cdot\frac{\cos 81^\circ}{\cos 27^\circ}
=\frac{\cos 81^\circ}{\cos 9^\circ}=\frac{\sin 9^\circ}{\cos 9^\circ} = \tg 9^\circ
\end{equation*}
болно.

\Problem
$a, b \geq 0,\hspace{0.3cm} 0<x<\frac{\pi}{2}$ байх бодит тоонуудын хувьд
\begin{equation*}
\left(1+\frac{a}{\sin x}\right)\left(1+\frac{b}{\cos x}\right)\geq \left(1+\sqrt{2ab}\right)^2
\end{equation*}
болохыг батал.

\TheSolution
Тэнцэтгэл бишийн хоёр талын хаалтуудыг задлан нийлбэр хэлбэрт бичвэл
\begin{equation*}
1+\frac{a}{\sin x} + \frac{b}{\cos x}+\frac{ab}{\sin x \cos x} \geq 1 + 2ab + 2\sqrt{2ab}
\end{equation*}
болно.
Кошийн тэнцэтгэл бишээр
\begin{equation*}
\frac{a}{\sin x}+\frac{b}{\cos x} \geq \frac{2\sqrt{ab}}{\sqrt{\sin x\cos x}}
\end{equation*}
Давхар өнцгийн томьёогоор $\sin x\cos x = \frac{1}{2}\sin 2x \leq \frac{1}{2}$ учир
\begin{equation*}
\frac{2\sqrt{ab}}{\sqrt{\sin x\cos x}} \geq 2\sqrt{2ab} \Rightarrow \frac{ab}{\sin x \cos x} \geq 2ab
\end{equation*}
болно. Сүүлийн 3 тэнцэтгэл бишийг ашиглавал
\begin{equation*}
1+\frac{a}{\sin x} + \frac{b}{\cos x}+\frac{ab}{\sin x \cos x} \geq 1 + 2ab + 2\sqrt{2ab}
\end{equation*}
болох нь батлагдана.

\Problem
$\triangle ABC$ - ны хувьд $\sin A + \sin B + \sin C \leq 1$ бол $min\left(A+B, B+C, C+A\right) < 30^\circ$ гэж батал.

\TheSolution
Бид $A \geq B \geq C$ гэе. Тэгвэл $B+C < 30^\circ$ гэж батлах шаардлагатай болно. Синусын теорем болон гурвалжны тэнцэтгэл бишийн $(b+c > a)$ чанар ёсоор $\sin B + \sin C > \sin A$ гэдгээс $\sin A + \sin B + \sin C > 2\sin A$ болно. Өгөгдсөн нөхцлийг ашиглавал $2\sin A < 1$ буюу $\sin A < \frac{1}{2}$ болно. $A$ өнцгийн хэмжээ гурвалжны бусад өнцгөөсөө их учир $A \geq \frac{A+B+C}{3} = 60^\circ$ болно. Иймд $A > 150^\circ$ учир $B+C < 30^\circ$ болох нь батлагдлаа.

\Problem
$ABC$ гурвалжны хувьд батал.
\begin{enumerate}[(a)]
\item
\begin{equation*}
\tg \frac{A}{2}\tg \frac{B}{2} + \tg \frac{B}{2} \tg \frac{C}{2} + \tg \frac{C}{2} \tg \frac{A}{2} = 1
\end{equation*}
\item
\begin{equation*}
\tg \frac{A}{2} \tg \frac{B}{2} \tg \frac{C}{2} \leq \frac{\sqrt{3}}{9}
\end{equation*}
\end{enumerate}

\TheSolution
\begin{enumerate}[(a)]
\item
Тангенсуудын нийлбэрийн томьёогоор
\begin{equation*}
\tg \frac{A}{2} + \tg \frac{B}{2} = \tg \frac{A+B}{2} \left(1-\tg \frac{A}{2}\tg \frac{B}{2}\right)
\end{equation*}
$A+B+C = 180^\circ$ тул $\frac{A+B}{2}=90^\circ - \frac{C}{2} \Rightarrow \tg\frac{A+B}{2} = \ctg\frac{C}{2}$ болно.
Иймд
\begin{align*}
\tg\frac{A}{2}\tg\frac{B}{2} + \tg\frac{B}{2}\tg\frac{C}{2}+\tg\frac{C}{2}\tg\frac{A}{2} =\\
=\tg\frac{A}{2}\tg\frac{B}{2} + \tg\frac{C}{2}\ctg\frac{C}{2}\left(1-\tg\frac{A}{2}\tg\frac{B}{2}\right) =\\
= \tg\frac{A}{2}\tg\frac{B}{2} + 1 - \tg\frac{A}{2}\tg\frac{B}{2} = 1
\end{align*}
болж батлагдлаа.


\item
Кошийн тэнцэтгэл биш ёсоор
\begin{align*}
1 = \tg \frac{A}{2}\tg\frac{B}{2} + \tg\frac{B}{2}\tg\frac{C}{2}+\tg\frac{C}{2}\tg\frac{A}{2} \geq\\
\geq 3\sqrt[3]{\left(\tg \frac{A}{2}\tg\frac{B}{2}\tg\frac{C}{2}\right)^2}
\end{align*}
Эндээс
\begin{equation*}
\tg\frac{A}{2}\tg\frac{B}{2}\tg\frac{C}{2} \leq \frac{\sqrt{3}}{9}
\end{equation*}
болох нь батлагдана.
\end{enumerate}

\Note
$(a)$ -ийн эквивалент хэлбэр нь 
\begin{equation*}
\ctg\frac{A}{2}+\ctg\frac{B}{2} + \ctg\frac{C}{2} = \ctg\frac{A}{2}\ctg\frac{B}{2}\ctg\frac{C}{2}
\end{equation*}
юм.

\Problem
Хурц өнцөгт $\triangle ABC$ -ны хувьд
\begin{enumerate}[(a)]
\item
$\tg A + \tg B + \tg C = \tg A\tg B \tg C$
\item
$\tg A \tg B\tg C \geq 3\sqrt{3}$
\end{enumerate}
болохыг батал.

\TheSolution
\begin{enumerate}[(a)]
\item
Тангенсуудын нийлбэрийн томьёо ашиглавал
\begin{align*}
\tg A +\tg B + \tg C = \tg (A+B)(1-\tg A \tg B) + \tg C = \tg(180^\circ - C)(1-\tg A \tg B) + \tg C =\\
= -\tg C + \tg A\tg B\tg C + \tg C = \tg A\tg B\tg C
\end{align*}
болно.
Өөрөөр хэлбэл 
$\tg A + \tg B + \tg C = \tg A\tg B \tg C$
болж батлагдлаа.
\item
Кошийн тэнцэтгэл биш ёсоор
\begin{equation*}
\tg A + \tg B + \tg C \geq 3\sqrt[3]{\tg A \tg B \tg C}
\end{equation*}
$(a)$ -г ашиглавал
\begin{equation*}
\tg A\tg B\tg C \geq 3\sqrt[3]{\tg A \tg B \tg C}
\end{equation*}
\begin{equation*}
\left(\tg A \tg B \tg C\right)^{\frac{2}{3}} \geq 3 \Rightarrow \tg A \tg B \tg C \geq 3\sqrt{3}
\end{equation*}
болно.
\end{enumerate}

\Note
$A+B+C = m\pi$ ба $A, B, C \neq \frac{k\pi}{2}$ байх $A, B, C$ өнцгүүдийн хувьд $(a)$ тэнцэтгэл биелнэ. Энд $k, m \in \mathbb{Z}$ болно.

\Problem
$\triangle ABC$ -ны хувьд $\ctg A\ctg B + \ctg B\ctg C + \ctg C\ctg A = 1$ гэж батал.

Эсрэгээр нь $xy+yz+zx = 1$ байх $x,y,z \in \mathbb{R}$ тоонуудын хувьд $\ctg A = x, \ctg B = y, \ctg C = z$ байх $\triangle ABC$ оршино гэж батал.

\TheSolution
$\triangle ABC$ -ныг тэгш өнцөгт гурвалжин ба $\measuredangle A = 90^\circ$ гэе. Тэгвэл $\ctg a = 0$ ба $B+C = 90^\circ$ гэдгээс $\ctg B \ctg C = 1$ болно. Өөрөөр хэлбэл тэгш өнцөгт гурвалжны хувьд тэнцэтгэл батлагдлаа. Одоо тэгш өнцөгт биш гурвалжны хувьд баталъя. Өгөгдсөн тэнцэтгэлийг $\tg A\tg B\tg C$ -гээр үржвэл
\begin{equation*}
\tg A + \tg B + \tg C = \tg A\tg B\tg C
\end{equation*}
болно. Дээрх тэнцэтгэл үнэн болохыг бид баталсан билээ. (20-р бодлого)

\Problem
$\triangle ABC$ -ны хувьд
\begin{equation*}
\sin^2 \frac{A}{2} + \sin^2 \frac{B}{2} + \sin^2 \frac{C}{2} + 2\sin \frac{A}{2}\sin \frac{B}{2} \sin \frac{C}{2} = 1
\end{equation*}
болохыг батал. Эсрэгээр нь $x, y, z < 1$ байх эерэг бодит тоонуудын хувьд 
\begin{equation*}
x^2+y^2+z^2+2xyz = 1
\end{equation*}
чанарыг хангаж байвал $x=\sin \frac{A}{2}, y = \sin \frac{B}{2}, z= \sin\frac{C}{2}$  байх $\triangle ABC$ оршино гэж батал.

\TheSolution
Хэрвээ бид өгөгдсөн хоёр дахь тэгшитгэлийн $x$-ийг квадрат тэгшитгэлийн шийд олох аргаар олвол
\begin{equation*}
x=\frac{-2yz + \sqrt{4y^2z^2-4(y^2+z^2-1)}}{2} = -yz + \sqrt{(1-y^2)(1-z^2)}
\end{equation*}
болно. Бид $y=\sin u, z=\sin v, 0^\circ < u, v < 90^\circ$ орлуулга хийвэл
\begin{equation*}
x=-\sin u \sin v + \cos u\cos v = \cos(u+v)
\end{equation*}
болно.

\Problem
$\triangle ABC$-ны хувьд батал.
\begin{enumerate}[(a)]
\item
\begin{equation*}
\sin \frac{A}{2}\sin \frac{B}{2}\sin \frac{C}{2} \leq \frac{1}{8}
\end{equation*}
\item
\begin{equation*}
\sin^2 \frac{A}{2} + \sin^2 \frac{B}{2} + \sin^2 \frac{C}{2} \geq \frac{3}{4}
\end{equation*}
\item
\begin{equation*}
\cos^2 \frac{A}{2}+\cos^2 \frac{B}{2}+\cos^2 \frac{C}{2} \leq \frac{9}{4}
\end{equation*}
\item
\begin{equation*}
\cos \frac{A}{2}\cos \frac{B}{2} \cos \frac{C}{2} \leq \frac{3\sqrt{3}}{8}
\end{equation*}
\item
\begin{equation*}
\csc \frac{A}{2} + \csc \frac{B}{2} + \csc \frac{C}{2} \geq 6
\end{equation*}
\end{enumerate}

\TheSolution
\begin{enumerate}[(a)]
\item
Бодлого 8 дээр бид 
\begin{equation*}
\sin \frac{A}{2} \sin \frac{B}{2} \sin \frac{C}{2} \leq \frac{abc}{(a+b)(b+c)(c+a)}
\end{equation*}
болохыг баталсан.

Кошийн тэнцэтгэл бишийг ашиглавал
\begin{equation*}
(a+b)(b+c)(c+a) \geq (2\sqrt{ab})(2\sqrt{bc})(2\sqrt{ca}) = 8abc
\end{equation*}
болно. Үүнийг орлуулвал
\begin{equation*}
\sin \frac{A}{2} \sin \frac{B}{2} \sin \frac{C}{2} \leq \frac{1}{8}
\end{equation*}
болно.
\item
(a) хэсгийн тэнцэтгэлийг бодлого 22 дээрх
\begin{equation*}
\sin^2 \frac{A}{2} + \sin^2 \frac{B}{2} + \sin^2 \frac{C}{2} + 2\sin \frac{A}{2}\sin \frac{B}{2} \sin \frac{C}{2} = 1
\end{equation*}
тэнцэтгэлтэй ашиглавал
\begin{equation*}
\sin^2 \frac{A}{2} + \sin^2 \frac{B}{2} + \sin^2 \frac{C}{2} \geq 1-\frac{1}{4} = \frac{3}{4}
\end{equation*}
болж батлагдлаа.
\item
(b) хэсгийн тэнцэтгэл бишд $\sin^2 x = 1-\cos^2 x$ орлуулга хийвэл
\begin{align*}
3 - \cos^2 \frac{A}{2} - \cos^2 \frac{B}{2} - \cos^2 \frac{C}{2} \geq \frac{3}{4} \\
\cos^2 \frac{A}{2} - \cos^2 \frac{B}{2} - \cos^2 \frac{C}{2} \leq 3- \frac{3}{4}\\
\cos^2 \frac{A}{2} - \cos^2 \frac{B}{2} - \cos^2 \frac{C}{2} \leq \frac{9}{4}
\end{align*}
болно.
\item
(d) хэсгийн тэнцэтгэл бишд кошийн тэнцэтгэл биш ашиглавал
\begin{align*}
\cos^2 \frac{A}{2} + \cos^2 \frac{B}{2} + \cos^2 \frac{C}{2} \geq 3\sqrt[3]{\cos^2 \frac{A}{2}\cos^2 \frac{B}{2} \cos^2 \frac{C}{2}}\\
3\sqrt[3]{\cos^2 \frac{A}{2}\cos^2 \frac{B}{2} \cos^2 \frac{C}{2}} \leq \frac{9}{4} \\
\cos \frac{A}{2} \cos \frac{B}{2} \cos \frac{C}{2} \leq \frac{3\sqrt{3}}{8}
\end{align*}
болно.
\item
8-р бодлогыг ашиглавал $\csc \frac{A}{2} \geq \frac{b+c}{a} = \frac{b}{a} + \frac{c}{a}$ болно. Үүнтэй адилаар $\csc \frac{B}{2} \geq \frac{a}{b} + \frac{c}{b}$ ба $\csc \frac{C}{2} \geq \frac{a}{c}+\frac{b}{c}$ байна. Кошийн тэнцэтгэл биш ашиглавал
\begin{align*}
\csc \frac{A}{2} + \csc \frac{B}{2} + \csc \frac{C}{2} \geq \frac{b}{a} + \frac{c}{a} + \frac{a}{b} + \frac{c}{b} + \frac{a}{c} + \frac{b}{c} \geq\\
\geq 6\sqrt[6]{\frac{b}{a}\frac{c}{a}\frac{a}{b}\frac{c}{b}\frac{a}{c}\frac{b}{c}} = 6
\end{align*}
болж батлагдав.
\end{enumerate}
\Note
(a) -г өөр аргаар баталъя. Бодлогын нөхцлөөр $\sin \frac{A}{2}, \sin \frac{B}{2}, \sin \frac{C}{2}$ бүгд эерэг тоонууд. $t=\sqrt[3]{\sin \frac{A}{2}\sin \frac{B}{2}\sin \frac{C}{2}}$ гэе. Тэгвэл бид $t \leq \frac{1}{2}$ гэж батлахад хангалттай. Кошийн тэнцэтгэл бишээр
\begin{align*}
\sin^2 \frac{A}{2} + \sin^2 \frac{B}{2} + \sin^2 \frac{C}{2} \geq 3t^2
\end{align*}
22-р бодлогод орлуулвал $3t^2 + 2t^3 \leq 1$ болно. Иймд
\begin{align*}
2t^3 + 3t^2 - 1 \leq 0\\
(t+1)(2t^2 + t -1) \leq 0\\
(t+1)^2(2t-1) \leq 0
\end{align*}
Эндээс харвал $t \leq \frac{1}{2}$ болж $(a)$ батлагдлаа.

\Problem
$\triangle ABC$ -ны хувьд
\begin{enumerate}[(a)]
\item
\begin{equation*}
\sin 2A + \sin 2B + \sin 2C = 4\sin A \sin B\sin C
\end{equation*}
\item
\begin{equation*}
\cos 2A + \cos 2B + \cos 2C = -1 - 4\cos A\cos B\cos C
\end{equation*}
\item
\begin{equation*}
\sin^2 A + \sin^2 B + \sin^2 C = 2+ 2\cos A \cos B \cos C
\end{equation*}
\item
\begin{equation*}
\cos^2 A + \cos^2 B + \cos^2 C + 2\cos A \cos B \cos C = 1
\end{equation*}
болохыг батал.

Эсрэгээр нь $0< x, y, z < 1$ эерэг бодит тоонууд
\begin{align*}
x^2 + y^2 + z^2 + 2xyz = 1
\end{align*} 
чанарыг хангаж байвал $x=\cos A, y= \cos B, z=\cos C$ байх хурц өнцөгт $\triangle ABC$ оршино гэж харуул.
\end{enumerate}

\TheSolution
$\cos 2x = 1-2\sin^2 x = 2\cos^2 x - 1$ ашиглавал (c) болон  (d) -гийн баталгаа (b)-гээс хялбархан мөрдөн гарна. Иймд бид (a) болон (b) -гийн баталгааг харуулъя.
\begin{enumerate}[(a)]
\item
Синусуудын нийлбэрийн томьёо болон $A+B+C = 180^\circ$ тэнцэтгэлийг ашиглавал
\begin{align*}
\sin 2A + \sin 2B + \sin 2C = 2\sin (A+B)\cos(A-B) + \sin 2C = \\
= 2\sin C \cos(A-B) + 2\sin C \cos C = 2\sin C \left[\cos (A-B) - \cos (A+B)\right] = \\
= 2\sin C \cdot \left[-2\sin A \sin (-B)\right] = 4\sin A\sin B\sin C
\end{align*}
болж батлагдлаа.
\item
Косинусуудын нийлбэрийн томьёо болон $A+B+C = 180^\circ$  тэнцэтгэлийг ашиглавал
\begin{align*}
\cos 2A + \cos 2B + \cos 2C = 2\cos(A+B)\cos(A-B) + \cos^2 C -1 = \\
= -2\cos C \cos(A-B) + \cos^2 C - 1 = -2\cos C(\cos(A-B) - \cos C) - 1 =\\
= -2\cos C(\cos(A-B) + \cos(A+B)) - 1 = - 4\cos A \cos B\cos C -1
\end{align*}
болж батлагдлаа.
\end{enumerate}
\Note
(d) -гийн сонирхолтой баталгааг харуулъя. Доорх систем тэгшитгэлийг авч үзье.
\begin{align*}
-x + (\cos B)y +(\cos C)z &= 0\\
(\cos B)x -y + (\cos A)z &= 0\\
(\cos C) x + (\cos A) y - z &= 0
\end{align*}
Тригонометрийн нийлбэр ялгаварын томъёог ашиглавал уг систем тэгшитгэлийн тэгээс ялгаатай шийд нь $(x, y, z) = (\sin A, \sin C, \sin B)$ болохыг хялбархан олж болно. Иймд уг системийн тодорхойлогч нь тэг болно. Өөрөөр хэлбэл
\begin{equation*}
\begin{vmatrix}
-1 & \cos B & \cos C\\
\cos B & -1 & \cos A\\
\cos C & \cos A & -1
\end{vmatrix}
=-1+2\cos A\cos B \cos C + \cos^2 A + \cos^2 B + \cos^2 C = 0
\end{equation*}
болж батлагдана.


\Problem
$\triangle ABC$-ны хувьд
\begin{enumerate}[(a)]
\item
$4R = \frac{abc}{[ABC]}$
\item
$2R^2\sin A\sin B\sin C = [ABC]$
\item
$2R^2\sin A\sin B\sin C = r(\sin A + \sin B + \sin C)$
\item
$r=4R\sin \frac{A}{2}\sin \frac{B}{2} \sin \frac{C}{2}$
\item
$a\cos A + b\cos B + c\cos C = \frac{abc}{2R^2}$
\end{enumerate}

\TheSolution
\begin{enumerate}[(a)]
\item
Синусын өргөтгөсөн теоремоор
\begin{equation*}
R= \frac{a}{2\sin A} = \frac{abc}{2bc\sin A} = \frac{abc}{4[ABC]}
\end{equation*}
\item
\begin{equation*}
2R^2\sin A\sin B\sin C = \frac{1}{2}(2R\sin A)(2R\sin B)(\sin C) = \frac{1}{2}ab\sin C = [ABC]
\end{equation*}
\item
$2R^2\sin A\sin B\sin C$-г авч үзье.
\begin{equation*}
2R^2\sin A \sin B\sin C = [ABC] = \frac{1}{2}bc\sin A = \frac{a+b+c}{2}\cdot r
\end{equation*}
болно. Синусын өргөтгөсөн теорем ашиглавал
\begin{equation*}
2R^2\sin A\sin B\sin C = rR(\sin A + \sin B + \sin C)
\end{equation*}
болно.
\item
Косинусын теоремоор
\begin{equation*}
\cos A = \frac{b^2+c^2-a^2}{2bc}
\end{equation*}
байна. Хагас өнцгийн теоремоор
\begin{align*}
\sin^2 \frac{A}{2} = \frac{1-\cos A}{2} = \frac{1}{2} - \frac{b^2 + c^2 -a^2}{4bc} = \frac{a^2-(b^2+c^2-2bc)}{4bc} =\\
= \frac{a^2 -(b-c)^2}{4bc} = \frac{(a-b+c)(a+b-c)}{4bc} = \\
= \frac{(2p-2b)(2p-2c)}{4bc} = \frac{(p-b)(p+b)}{bc}
\end{align*}
болно. Энд $p=\frac{a+b+c}{2}$ болно. Өөрөөр хэлбэл
\begin{equation*}
\sin \frac{A}{2} = \sqrt{\frac{(p-b)(p+b)}{bc}}
\end{equation*}
болно. Нөгөө 2 өнцөг дээр мөн адилаар томьёог олон үржвэрийг олвол
\begin{equation*}
\sin \frac{A}{2}\sin \frac{B}{2} \sin \frac{C}{2} = \frac{(p-a)(p-b)(p-c)}{abc} = \frac{p(p-a)(p-b)(p-c)}{pabc}
\end{equation*}
болно. Героны томьёогоор
\begin{equation*}
\sin \frac{A}{2}\sin \frac{B}{2} \sin \frac{C}{2} = \frac{[ABC]^2}{pabc}
\end{equation*}
болох ба
\begin{equation*}
\sin \frac{A}{2}\sin \frac{B}{2} \sin \frac{C}{2} = \frac{[ABC]}{p}\frac{[ABC]}{abc} = r\cdot \frac{1}{4R}
\end{equation*}
болж батлагдлаа.
\item
Синусын өргөтгөсөн теоремоор  $a\cos A = 2R\sin A\cdot\cos A = R\sin 2A$ болно. Үүнтэй адилаар $b\cos B = R\sin 2B$ болох ба $c\cos C = R\sin 2C$ болно.
(a) болон (b) хоёрыг ашиглавал
\begin{equation*}
4R\sin A\sin B\sin C = \frac{abc}{2R^2}
\end{equation*}
болно. Иймд одоо бид 
\begin{equation*}
\sin 2A + \sin 2B + \sin 2C = 4\sin A\sin B\sin C
\end{equation*}
 гэж батлахад хангалттай болно. Бид дээрх тэнцэтгэлийг 24-р бодлогоор батласан билээ.
\end{enumerate}

\Problem
$\triangle ABC$-ны хагас периметрийг $p$ гэе. Тэгвэл
\begin{enumerate}[(a)]
\item
\begin{equation*}
p=4R\cos\frac{A}{2}\cos\frac{B}{2}\cos\frac{C}{2}
\end{equation*}
\item
\begin{equation*}
p \leq \frac{3\sqrt{3}}{2}R
\end{equation*}
\end{enumerate}
болохыг батал.

\TheSolution
\begin{enumerate}[(a)]
\item
Бид $p = \frac{[ABC]}{r}$ томъёог мэдэх билээ. Уг томьёонд 25-р бодлогын (b) болон (d)-г орлуулвал
\begin{equation*}
p=\frac{R\sin A\sin B\sin C}{2\sin \frac{A}{2}\sin\frac{B}{2}\sin \frac{C}{2}}
\end{equation*}
болно. Давхар өнцгийн теорем ашиглавал
\begin{equation*}
p =4R\cos\frac{A}{2}\cos \frac{B}{2}\cos\frac{C}{2}
\end{equation*}
болж батлагдлаа.
\item
23-р бодлогын (d) хэсэг болон дээрх тэнцэтгэлийг ашиглан батлана.
\end{enumerate}

\Problem
$\triangle ABC$-ны хувьд
\begin{enumerate}[(a)]
\item
\begin{equation*}
\cos A + \cos B + \cos C = 1 + 4\sin \frac{A}{2}\sin \frac{B}{2}\sin \frac{C}{2}
\end{equation*}
\item
\begin{equation*}
\cos A + \cos B + \cos C \leq \frac{3}{2}
\end{equation*}
болохыг харуул.
\end{enumerate}

\TheSolution
\begin{equation*}
\cos A + \cos B = 2\cos \frac{A+B}{2}\cos \frac{A-B}{2} = 2\sin \frac{C}{2}\cos \frac{A-B}{2}
\end{equation*}
\begin{equation*}
1-\cos C = 2\sin^2 \frac{C}{2} = 2\sin \frac{C}{2}\cos \frac{A+B}{2}
\end{equation*}
эдгээрийг тэнцэтгэлдээ орлуулвал
\begin{align*}
&\cos A + \cos B + \cos C = 1 + 4\sin \frac{A}{2}\sin \frac{B}{2}\sin \frac{C}{2}\\
&\cos A + \cos B + \cos C -1 = 4\sin \frac{A}{2}\sin \frac{B}{2}\sin \frac{C}{2}\\
&2\sin \frac{C}{2}\cos \frac{A-B}{2} - 2\sin \frac{C}{2}\cos \frac{A+B}{2} = 4\sin \frac{A}{2}\sin \frac{B}{2}\sin \frac{C}{2}\\
&2\sin \frac{C}{2} \left[\cos \frac{A-B}{2} - \cos \frac{A+B}{2}\right] = 4\sin \frac{A}{2}\sin \frac{B}{2}\sin \frac{C}{2}\\
&2\sin \frac{C}{2} \cdot 2\sin \frac{A}{2} \sin \frac{B}{2} = 4\sin \frac{A}{2}\sin \frac{B}{2}\sin \frac{C}{2}
\end{align*}
болж (a) батлагдлаа.

25 дугаар бодлогын (c) -г ашиглавал
\begin{equation*}
\cos A + \cos B + \cos C = 1+\frac{r}{R}
\end{equation*}
болно.

Гурвалжны багтаасан ба гурвалжинд багтсан тойргийн төвүүд болон радиусиудыг харгалзан $O, I, R, r$ гэвэл эйлерийн теорем ёсоор $|OI|^2 = R^2 - 2Rr$ болно. $|OI|^2 \geq 0$ учир $\frac{r}{R} \leq \frac{1}{2}$ болно. Үүнийг дээрх тэнцэтгэлд орлуулвал (b) батлагдана.

\Problem
$\triangle ABC$-ны хувьд
\begin{enumerate}[(a)]
\item
\begin{equation*}
\cos A \cos B \cos C \leq \frac{1}{8}
\end{equation*}
\item
\begin{equation*}
\sin A \sin B \sin C \leq \frac{3\sqrt{3}}{8}
\end{equation*}
\item
\begin{equation*}
\sin A + \sin B + \sin C \leq \frac{3\sqrt{3}}{2}
\end{equation*}
\item
\begin{equation*}
\cos^2 A + \cos^2 B + \cos^2 C \geq \frac{3}{4}
\end{equation*}
\item
\begin{equation*}
\sin^2 A + \sin^2 B + \sin^2 C \leq \frac{9}{4}
\end{equation*}
\item
\begin{equation*}
\cos 2A + \cos 2B + \cos 2C \geq -\frac{3}{2}
\end{equation*}
\item
\begin{equation*}
\sin 2A + \sin 2B + \sin 2C \leq \frac{3\sqrt{3}}{2}
\end{equation*}
\end{enumerate}
болохыг батал.

\TheSolution
(a) -гийн хувьд $\triangle ABC$ нь мохоо өнцөгт гурвалжин байвал тэнцэтгэл бишийн зүүн гар тал сөрөг болох тул тэнцэтгэл биш биелнэ.
$t=\sqrt[3]{\cos \frac{A}{2}\cos \frac{B}{2}\cos \frac{C}{2}}$ гэвэл $t\leq \frac{1}{2}$ гэж батлахад хангалттай болно. Кошийн тэнцэтгэл биш ёсоор
\begin{equation*}
\cos^2 \frac{A}{2} + \cos^2 \frac{B}{2} + \cos^2 \frac{C}{2} \geq 3t^2
\end{equation*}
болж 22 дугаар бодлогын тэмдэглэлтэй адилаар батлагдана. (d) -г батлахад 24 дүгээр бодлогын (d)-г ашиглавал 
\begin{equation*}
\cos^2 A + \cos^2 B + \cos^2 C = 1- 2\cos A \cos B \cos C \geq \frac{3}{4}
\end{equation*}
болно. Тригонометрийн үндсэн адилтгалаар (d) болон (e) тэнцэтгэл бишүүд нь эквивалент юм.

(e) тэнцэтгэл бишийн хувьд кошийн тэнцэтгэл бишээр
\begin{equation*}
\frac{9}{4} \geq \sin^2 A + \sin^2 B + \sin^2 C \geq 3\sqrt[3]{\sin^2 A \sin^2 B \sin^2 C}
\end{equation*}
\begin{equation*}
\sin A\sin B \sin C \leq \frac{3\sqrt{3}}{8}
\end{equation*}
болж (b) батлагдана.

$(a-b)^2 + (b-c)^2 + (c-a)^2 \geq 0 \Rightarrow 3(a^2+b^2 + c^2) \geq (a+b+c)^2$  болно. $a=\sin A, b=\sin B, c= \sin C$ гэвэл
\begin{equation*}
3(\sin^2 A + \sin^2 B + \sin^2 C) \geq (\sin A + \sin B + \sin C)^2
\end{equation*}
(e) -г орлуулвал
\begin{equation*}
\frac{27}{8} \geq (\sin A + \sin B + \sin C)^2 \Rightarrow \sin A + \sin B + \sin C \leq \frac{3\sqrt{3}}{2}
\end{equation*}
болж (c) батлагдлаа. (f) нь (e) болон $\cos 2x = 2\cos^2 x -1$ адилтгалыг ашиглан батлана. (g) тэнцэтгэл бишийг (b) тэнцэтгэл биш болон $\sin 2A + \sin 2B + \sin 2C = 4\sin A\sin B \sin C$ адилтгалыг ашиглан батална. Сүүлийн тэнцэтгэл үнэн болохыг 25 дугаар бодлогоор харуулсан билээ.


\Problem
$x \neq \frac{k\pi}{6}, (k \in \mathbb{Z})$ бол 
\begin{equation*}
\frac{\tg 3x}{\tg x} = \tg \left(\frac{\pi}{3}-x \right)\tg \left(\frac{\pi}{3} + x\right)
\end{equation*}
болохыг батал.

\TheSolution
Гурван давхар өнцгийн томъёогоор
\begin{align*}
\tg 3x = \frac{3\tg x - \tg^3 x}{1-3\tg^2 x} = \tg x \cdot \frac{(\sqrt{3}-\tg x)(\sqrt{3}+\tg x)}{(1-\sqrt{3}\tg x)(1+\sqrt{3}\tg x)} = \\
= \frac{\sqrt{3}-\tg x}{1+\sqrt{3}\tg x}\cdot\tg x\cdot \frac{\sqrt{3}+\tg x}{1-\sqrt{3}\tg x} =
\tg \left(\frac{\pi}{3}-x\right)\tg x \tg \left(\frac{\pi}{3} + x\right)
\end{align*}
болж батлагдлаа.

\Problem
[AMC12P 2002]
\begin{equation*}
(1+\tg 1^\circ)(1+\tg 2^\circ)\cdots(1+\tg 45^\circ) = 2^n
\end{equation*}
байх $n$-г ол.

\ASolution
\begin{align*}
1+\tg k^\circ = 1+\frac{\sin k^\circ}{\cos k^\circ} = \frac{\cos k^\circ + \sin k^\circ}{\cos k^\circ} = \\
= \frac{\sqrt{2\sin(45^\circ + k^\circ)}}{\cos k^\circ} = \frac{\sqrt{2}\cos(45^\circ-k^\circ)}{\cos k^\circ}
\end{align*}
Иймээс
\begin{align*}
(1+\tg k^\circ)(1+\tg(45^\circ-k^\circ)) = \frac{\sqrt{2}\cos(45^\circ - k^\circ)}{\cos k^\circ}\cdot \frac{\sqrt{2}\cos k^\circ}{\cos (45^\circ - k^\circ)} = 2
\end{align*}
болох ба цаашилбал
\begin{align*}
(1+\tg 1^\circ)(1+\tg 2^\circ)\cdots(1+\tg 45^\circ) =\\
(1+\tg 1^\circ)(1+\tg 44^\circ)(1+\tg 2^\circ)(1+\tg 43^\circ)\\
\cdots(1+\tg 22^\circ)(1+\tg 23^\circ)(1+\tg 45^\circ) = 2^{23}
\end{align*}
Иймд $n=23$ болно.

\ASolution
\begin{align*}
(1+\tg k^\circ)(1+\tg (45^\circ-k^\circ)) = 1+\left[\tg k^\circ + \tg (45^\circ - k^\circ)\right] + \tg k^\circ \tg (45^\circ - k^\circ) = \\
= 1+\tg 45^\circ\left[1- \tg k^\circ \tg(45^\circ - k^\circ)\right] + \tg k^\circ \tg (45^\circ - k^\circ) = 2
\end{align*}
болох ба
\begin{align*}
(1+\tg 1^\circ)(1+\tg 2^\circ)\cdots(1+\tg 45^\circ) =\\
(1+\tg 1^\circ)(1+\tg 44^\circ)(1+\tg 2^\circ)(1+\tg 43^\circ)\\
\cdots(1+\tg 22^\circ)(1+\tg 23^\circ)(1+\tg 45^\circ) = 2^{23}
\end{align*}
Иймд $n=23$ болно.

\Problem
[AIME 2003] Координатын хавтгайд $A=(0, 0); B=(b, 2)$ цэгүүд өгчээ. $ABCDEF$ зөв зургаан өнцөгтийн $\measuredangle FAB = 120^\circ, AB \| DE, BC\|EF, CD\|FA$ байх ба оройнуудын $y$ тэнхлэг дээрх координатын олонлог нь ${0, 2, 4, 6, 8}$ болно. Зургаан өнцөгтийн талбай $m\sqrt{n}$ бол $m+n$-г ол. Энд $m, n > 0$ ба $n$ нь ямар ч анхны тооны квадратад хуваагдахгүй болно.

\Note

\Problem
Тооны машин дээрх урвууг олдог товчлуур эвдэрсэн гэж саная. Тэгвэл тригонометрийн $\sin, \cos, \tg, \arcsin, \arccos, \arctg$ товчлууруудыг ашиглан аливаа тооны урвууг олж болохыг харуул.

\TheSolution
$0<\theta < \pi/2$ өнцгийн хувьд $\arccos\sin \theta = \pi/2 - \theta$ ба $\tg\left(\frac{\pi}{2} - \theta\right) = \frac{1}{\tg\theta}$ байна. $\tg\theta$ функцийн утгын муж нь $E(\tg\theta): [0, \inf[$ байх тул дурын $x > 0$ тооны хувьд
\begin{equation*}
\tg\arccos\sin\arctg x = \tg \left(\frac{\pi}{2} - \arctg x \right) = \frac{1}{x}
\end{equation*}
болно. Уг бодлогын өөр нэг хариу нь $\tg\arcsin\cos\arctg$ болно.

\Problem
$\triangle ABC$ -ны хувьд $A-B=120^\circ$ ба $R=8r$ бол $C$-г ол.

\TheSolution
25-р бодлогын (d)-г ашиглавал
\begin{equation*}
2\sin\frac{A}{2}\sin\frac{B}{2}\sin\frac{C}{2} = \frac{1}{16}
\end{equation*}
болох ба синусуудын нийлбэрийн томъёогоор
\begin{equation*}
\left(\cos\frac{A-B}{2} - \cos\frac{A+B}{2}\right)\sin\frac{C}{2} = \frac{1}{16}
\end{equation*}
болно. $A-B=120^\circ$ болохыг тооцвол
\begin{align*}
\left(\frac{1}{2} - \sin\frac{C}{2}\right)\sin\frac{C}{2} = \frac{1}{16} \\
\left(\frac{1}{4} - \sin\frac{C}{2}\right)^2 = 0
\end{align*}
болно. Эндээс үзвэл $\sin\frac{C}{2} = \frac{1}{4}$ болох ба $\cos C = 1-2\sin^2 \frac{C}{2} = \frac{7}{8}$ болно.

\Problem
$\triangle ABC$ -ны хувьд
\begin{equation*}
\frac{a-b}{a+b} = \tg\frac{A-B}{2}\tg\frac{C}{2}
\end{equation*}
батал.

\TheSolution
Синусын теорем болон синусуудын ялгаварын томъёог хэрэглэвэл
\begin{align*}
\frac{a-b}{a+b} = \frac{\sin A - \sin B}{\sin A + \sin B} = \frac{2\sin\frac{A-B}{2}\cos\frac{A+B}{2}}{2\sin\frac{A+B}{2}\cos\frac{A-B}{2}} = \\
= \tg\frac{A-B}{2}\ctg\frac{A+B}{2} = \tg\frac{A-B}{2}\tg\frac{C}{2}
\end{align*}
болно.

\Problem
$\triangle ABC$-ны хувьд $\frac{a}{b} = 2+\sqrt{3}$ ба $C = 60^\circ$ бол $A$ болон $B$ өнцгийн хэмжээг ол.

\TheSolution
Өмнөх бодлогын үр дүнг ашиглавал
\begin{equation*}
\frac{\frac{a}{b} - 1}{\frac{a}{b} + 1} = \tg \frac{A-B}{2}\tg\frac{C}{2}
\end{equation*}
болно. Өөрөөр хэлбэл
\begin{equation*}
\frac{1+\sqrt{3}}{\sqrt{3}+3} = \tg\frac{A-B}{2}\cdot \frac{1}{\sqrt{3}}
\end{equation*}
гэдгээс $\tg\frac{A-B}{2} = 1$ болно. Иймд $A-B = 90^\circ$ болох ба $A+B = 180^\circ - C = 120^\circ$ тул $A=105^\circ$ ба $B=15^\circ$ болно.

\Problem
$a, b, c$ нь $-1$ болон $1$-ээс ялгаатай бодит тоонууд ба $a+b+c = abc$ бол
\begin{equation*}
\frac{a}{1-a^2} + \frac{b}{1-b^2} + \frac{c}{1-c^2} = \frac{4abc}{(1-a^2)(1-b^2)(1-c^2)}
\end{equation*}
болохыг батал.

\TheSolution
Бид 20-р бодлогын \textbf{Тэмдэглэл}-ийг эргэн харья. $A+B+C = m\pi$ ба $A, B, C \neq \frac{k\pi}{2}$ байх $A, B, C$ өнцгүүдийн хувьд
\begin{equation*}
\tg A + \tg B + \tg C  = \tg A \tg B \tg C
\end{equation*}
гэсэн тэнцэтгэл биелнэ. Энд $m, k \in \mathbb{Z}$ болно.

$a=\tg x, b = \tg y, c=\tg z$ гэе. Тэгвэл бодлогын $a+b+c = abc$ нөхцлөөс $\tg(x+y+z) = 0$ болно. Эндээс давхар өнцгийн тангенсийн томьёогоор
\begin{equation*}
\tg (2x+2y+2z) = \frac{2\tg (x+y+z)}{1-\tg^2 (x+y+z)} = 0
\end{equation*}
болох тул
\begin{equation*}
\tg 2x + \tg 2y + \tg 2z = \tg 2x\tg 2y\tg 2z
\end{equation*}
болно. Давхар өнцгийн тангенсийн томьёогоор задалж бичвэл
\begin{equation*}
\frac{2\tg x}{1-\tg^2 x} + \frac{2\tg y}{1-\tg^2 y} + \frac{2\tg z}{1-\tg^2 z} = \frac{2\tg x}{1-\tg^2 x}\cdot           \frac{2\tg y}{1-\tg^2 y}\cdot \frac{2\tg z}{1-\tg^2 z}
\end{equation*}
болох ба $a, b, c$ -г орлуулвал
\begin{equation*}
\frac{a}{1-a^2} + \frac{b}{1-b^2} + \frac{c}{1-c^2} = \frac{4abc}{(1-a^2)(1-b^2)(1-c^2)}
\end{equation*}
болно.

\Problem
$\triangle ABC$ гурвалжин адил хажуут байх зайлшгүй бөгөөд хүрэлцээтэй нөхцөл нь
\begin{equation*}
a\cos B + b\cos C + c\cos  A = \frac{a+b+c}{2}
\end{equation*}
болохыг батал.

\TheSolution
Синусын өргөтгөсөн теоремоор $a=2R\sin A, b = 2R\sin B, c = 2R\sin C$ байна. Тэгвэл дээрх тэнцэтгэл
\begin{align*}
2\sin A\cos B + 2\sin B\cos C + 2\sin C\cos A = \sin A + \sin B + \sin C\\
\sin(A+B) + \sin(A-B) + \sin(B+C) + \sin(B-C) + \sin(C+A) + \sin(C-A) = \sin A + \sin B + \sin C
\end{align*}
болно. Энд $A+B+C = 180^\circ$ гэдгээс $\sin (A+B) = \sin C$, $\sin (B+C) = \sin A$, $\sin (C+A) = \sin B$ болох тул
\begin{equation*}
\sin (A-B) + \sin (B-C) + \sin (C-A) = 0
\end{equation*}
болно. 15-р бодлогын үр дүнг ашиглавал
\begin{equation*}
4\sin\frac{A-B}{2} \sin\frac{B-C}{2} \sin\frac{C-A}{2} = 0
\end{equation*}
болно. Эндээс $\triangle ABC$ адил хажуут гурвалжин байх зайлшгүй бөгөөд хүрэлцээтэй нөхцөл нь
\begin{equation*}
a\cos B + b\cos C + c\cos  A = \frac{a+b+c}{2}
\end{equation*}
болох нь батлагдлаа.

\Problem
$a=\frac{2\pi}{1999}$ бол дараах илэрхийллийн утгыг тооцоол.
\begin{equation*}
\cos a\cos 2a\cos 3a \cdots \cos 999a
\end{equation*}

\TheSolution
Бидний олох илэрхийллийн утгыг $P$ гээд $Q = \sin a\sin 2a \sin 3a  \cdots \sin 999a$ гэе. Тэгвэл
\begin{align*}
2^999 PQ = (2\sin a\cos a)(2\sin 2a\cos 2a)\cdots(2\sin 999a\cos 999a) = \\
= \sin 2a\sin 4a\sin 6a \cdots \sin 1998a = \\
= (\sin 2a\sin 4a\sin 6a \cdots \sin 998a)\left[-\sin(2\pi-1000a)\right]\cdot \left[-\sin(2\pi-1002a)\right]\cdot\\
\cdot \left[-\sin(2\pi-1004a)\right]\cdots \left[-\sin(2\pi-1998a)\right] = \\
 = \sin 2a\sin 4a \cdots \sin 998a \sin 999a \sin 997a \cdots \sin a = Q
\end{align*}
$Q \neq 0$ гэдэг нь ойлгомжтой тул $P = \frac{1}{2^999}$ болно.

\Problem
$\alpha, \beta \neq \frac{k\pi}{2}$ ба $k \in \mathbb{Z}$ бол
\begin{equation*}
\frac{\sec^4 \alpha}{\tg^2 \beta} + \frac{\sec^4 \beta}{\tg^2 \alpha}
\end{equation*}
илэрхийллийн хамгийн бага утгыг ол.

\TheSolution
$a=\tg^ \alpha, b = \tg^2 \alpha$ гэе. Тэгвэл $a, b > 0$ тоонуудын хувьд
\begin{equation*}
\frac{(a+1)^2}{b} + \frac{(b+1)^2}{a}
\end{equation*}
илэрхийллийн хамгийн бага утгыг олоход хангалттай болно.
\begin{align*}
\frac{(a+1)^2}{b} + \frac{(b+1)^2}{a} = \frac{a^2+2a+1}{b} + \frac{b^2 + 2b + 1}{a} = \\
 = \left(\frac{a^2}{b} + \frac{1}{b} + \frac{b^2}{a} + \frac{1}{a}\right) + 2\left(\frac{a}{b} + \frac{b}{a}\right)
\end{align*}
болно. Кошийн тэнцэтгэл бишээр
\begin{equation*}
\left(\frac{a^2}{b} + \frac{1}{b} + \frac{b^2}{a} + \frac{1}{a}\right) + 2\left(\frac{a}{b} + \frac{b}{a}\right) \geq
4\sqrt[4]{\frac{a^2}{b}\cdot \frac{1}{b}\cdot\frac{b^2}{a}\cdot \frac{1}{a}} + 4\sqrt{\frac{a}{b}\cdot \frac{b}{a}} = 8
\end{equation*}
болно. Дээрх тэнцэтгэл бишийн тэнцэх нөхцөл нь $a=b = 1$ болно. Өөрөөр хэлбэл $\alpha = \pm 45^\circ + k\cdot 180^\circ$, $\beta = \pm 45^\circ + k\cdot 180^\circ$, $k \in \mathbb{Z}$ үед өгөгдсөн илэрхийлэл хамгийн бага утгаа буюу $8$ гэсэн утга авна.

\Problem
$0<x<\frac{\pi}{2}$ бол
\begin{equation*}
\frac{(\sin x)^{2y}}{(\cos x)^{y^2/2}} + \frac{(\cos x)^{2y}}{(\sin)^{y^2/2}} = \sin 2x
\end{equation*}
тэгшитгэлийг хангах бүх $(x, y)$ хос шийдийг ол.

\TheSolution
Кошийн тэнцэтгэл бишээр
\begin{equation*}
\frac{(\sin x)^{2y}}{(\cos x)^{y^2/2}} + \frac{(\cos x)^{2y}}{(\sin)^{y^2/2}} \geq 2(\sin x\cos x)^{y-y^2/4}
\end{equation*}
гэдгээс
\begin{equation*}
2\sin x\cos x = \sin 2x \geq 2(\sin x\cos x)^{y-y^2/4}
\end{equation*}
болно. $\sin x\cos x < 1$ учир $1 \leq y - y^2/4 \Rightarrow (1-y/2)^2 \leq 0$ болно. Дээрх тэнцэтгэл бишүүдийн тэнцэх нөхцлийг авч үзвэл $y =2$ ба $\sin x = \cos x$ болох тул өгөгдсөн тэгшитгэлийн цор ганц хос шийд нь $(x, y) = \left(\frac{\pi}{4}, 2\right)$ болно.

\Problem
$\cos 1^\circ$ иррационал тоо болохыг батал.

\TheSolution
Эсрэгээр нь $\cos 1^\circ$ рационал тоо гэе. Тэгвэл дурын $n \geq 1$  тооны хувьд
\begin{equation*}
\cos (n^\circ + 1^\circ) + \cos (n^\circ - 1^\circ) = 2\cos n^\circ \cos 1^\circ
\end{equation*}
тэнцэтгэл биелэх тул математик индукцын зарчим ёсоор $\cos 2^\circ$, $\cos 3^\circ \cdots$  тоонууд рационал тоо болно. Гэвч $\cos 30^\circ$ иррационал тоо болохыг бид мэдэх тул зөрчилд хүрнэ. Иймд $\cos 1^\circ$ рационал тоо биш болох нь батлагдлаа.

\Problem
[USAMO 2002 proposal by Cecil Rousseau] Хэрвээ $x_1^2 + x_2^2 = y_1^2 + y_2^2 = c^2$ бол
\begin{equation*}
S = (1-x_1)(1-y_1) + (1-x_2)(1-y_2)
\end{equation*}
илэрхийллийн хамгийн их утгыг ол.

\TheSolution
$x_1, x_2$ хоёрыг $P$ цэгийн координатууд гэвэл бодлогын нөхцлөөс $P$ цэг координатын эх дээр төвтэй $c$ радиустай тойрог дээр оршино. Иймд бид $x_1 = c\cos \theta$, $x_2 = c\sin \theta$ гэж илэрхийлж болно. Үүнтэй адилаар $y_1 = c\cos \phi$, $y_2 = c\sin \phi$ болно. Тэгвэл
\begin{align*}
S = 2-c(\cos \theta + \sin \theta + \cos \phi + \sin \phi) + c^2 (\cos \theta\cos \phi + \sin \theta \sin \phi) = \\
 = 2 - \sqrt{2}c\left[\sin(\theta + \pi/4) + \sin(\phi + \pi/4)\right] + c^2 \cos(\theta - \phi) \leq \\
 \leq 2 + 2\sqrt{2}c + c^2 = (\sqrt{2} + c)^2
\end{align*}
болно. Тэнцэх нөхцөл нь $\theta = \phi = 5\pi/4$. Өөрөөр хэлбэл $x_1 = x_2 = y_1 = y_2 = \frac{-c\sqrt{2}}{2}$ үед өгөгдсөн илэрхийллийн утга хамгийн их буюу $S = (\sqrt{2}+ c)^2$ болно.

\Problem
Дурын $0<a, b < \frac{\pi}{2}$ тоонуудын хувьд
\begin{equation*}
\frac{\sin^3 a}{\sin b} + \frac{\cos^3 a}{\cos b} \geq \sec(a-b)
\end{equation*}
тэнцэтгэл бишийг батал.

\TheSolution
Тэнцэтгэл бишийн хоёр талыг $\sin a\sin b + \cos a\cos b = \cos(a-b)$ тэнцэтгэлээр үржвэл
\begin{equation*}
\left(\frac{\sin^3 a}{\sin b} + \frac{\cos^3 a}{\cos b}\right)\left(\sin a \sin b + \cos a \cos b\right) \geq 1
\end{equation*}
болно. Тэнцэтгэл бишийн зүүн талын хаалтыг задалж кошийн тэнцэтгэл биш ашиглавал
\begin{align*}
\sin^4 a + \frac{\cos^3 a \sin a \sin b}{\cos b} + \frac{\sin^3 a\cos a \cos b}{\sin b} + \cos^4 a \geq \\
\geq \sin^4 a + 2\sqrt{\cos^4 a\sin^4 a} + \cos^4 = (\sin^2 a + \cos^2 a)^2 = 1
\end{align*}
болж батлагдав.

\Problem
Хэрэв $\sin \alpha \cos \beta = -\frac{1}{2}$ бол $\cos \alpha \sin \beta$-гийн утгын мужийг ол.

\TheSolution
\begin{equation*}
\sin(\alpha + \beta) = \sin \alpha\cos \beta + \cos \alpha \sin \beta = -\frac{1}{2} + \cos \alpha \sin \beta
\end{equation*}
$-1 \leq \sin(\alpha + \beta) \leq 1$ байдаг тул $-\frac{1}{2}\leq \cos \alpha\sin \beta \leq \frac{3}{2}$ болно.
Үүнтэй адилаар
\begin{equation*}
\sin (\alpha - \beta) = \sin\alpha\cos\beta - \cos\alpha\sin \beta
\end{equation*}
болохыг тооцвол $-\frac{3}{2} \leq \cos\alpha\sin\beta \leq \frac{1}{2}$ болно. Дээрх үр дүнгүүдийг нэгтгэвэл
\begin{equation*}
-\frac{1}{2} \leq \cos\alpha\sin \beta \leq \frac{1}{2}
\end{equation*}
болно. Гэвч $\cos\alpha\sin \beta$ -гийн бүх утга $\left[-\frac{1}{2}; \frac{1}{2}\right]$ завсарт оршино гэж харуулах шаардлагатай.
\begin{align*}
(\cos \alpha\sin\beta)^2 &= (1-\sin^2\alpha)(1-\cos^2\beta) = \\
&= 1-(\sin^2\alpha + \cos^2\beta) + \sin^2 \alpha\cos^2\beta = \\
&= \frac{5}{4} - (\sin^2\alpha + \cos^2\beta) = \\
&= \frac{5}{4} - (\sin\alpha + \cos\beta)^2 + 2\sin\alpha\cos\beta = \\
&= \frac{1}{4} - (\sin\alpha + \cos\beta)^2
\end{align*}
$x=\sin \alpha, y = \cos \beta$ гэвэл $-1\leq x, y \leq 1$ ба $xy=-\frac{1}{2}$ болно. $x$ ба $y$ нийлбэрийн утгын мужийг олъё. Хэрэв $s=x+y$ гэвэл $x, y$ нь
\begin{equation*}
u^2 - su - \frac{1}{2} = 0
\end{equation*}
квадрат тэгшитгэлийн язгуурууд болох ба $\left\lbrace x, y\right\rbrace = \left\lbrace\frac{s+\sqrt{s^2 + 2}}{2}, \frac{s-\sqrt{s^2 + 2}}{2}\right\rbrace$ болно. $\frac{s+\sqrt{s^2 + 2}}{2} \leq 1$ нөхцлөөс $s \leq \frac{1}{2}$ гэдэг нь илэрхий. Үүнтэй адил нөхцлийг шалгаснаар $s$- ийн утгын муж нь $\left[-\frac{1}{2}, \frac{1}{2}\right]$ болох тул $s^2 \in \left[0, \frac{1}{4}\right]$ болно. Иймд $(\cos \alpha\sin \beta)^2 \in \left[0, \frac{1}{4}\right] \Rightarrow \cos\alpha\sin\beta \in \left[ -\frac{1}{2}, \frac{1}{2}\right]$ болно.

\Problem
$a, b, c$ бодит тоонуудын хувьд
\begin{equation*}
(ab+bc+ca -1)^2 \leq (a^2 +1)(b^2 + 1)(c^2 + 1)
\end{equation*}
болохыг батал.

\TheSolution
$a= \tg x, b = \tg y, c = \tg z$ ба $-\frac{\pi}{2}<x,y,z<\frac{\pi}{2}$ гэе. Тэгвэл $a^2 + 1 = \sec^2 x, b^2 + 1 = \sec^2 y, c^2 + 1 = \sec^2 z$ болно. Батлах тэнцэтгэл бишийн хоёр талыг $\cos^2 x\cos^2 y \cos^2 z$ -ээр үржүүлвэл
\begin{equation*}
\left[ (ab+bc+ca-1)\cos x\cos y \cos z\right]^2 \leq 1
\end{equation*}
болно.
\begin{align*}
(ab+bc)\cos x\cos y\cos z = \sin x\sin y\cos z + \sin y\sin z\cos x = \\
 = \sin y\sin (x+z)
\end{align*}
\begin{align*}
(ca - 1) \cos x \cos y\cos z = \sin z\sin x\cos y - \cos x \cos y \cos z = \\
= -\cos y \cos(x+z)
\end{align*}
гэдгийг орлуулвал
\begin{align*}
[(ab+bc+ca - 1)\cos x\cos y\cos z]^2 = \\
= [\sin y \sin(x+z) - \cos y \cos(x+z)]^2 = \\
 = \cos^2(x+y+z) \leq 1
\end{align*}
болж батлагдлаа.


\Problem
\begin{equation*}
(\sin x + a\cos x)(\sin x + b\cos x) \leq 1 + \left(\frac{a+b}{2}\right)^2
\end{equation*}
болохыг батал.

\TheSolution
Хэрвээ $\cos x = 0$ бол дээрх тэнцэтгэл биш $\sin^2 x \leq 1+ \left(\frac{a+b}{2}\right)^2$ болж биелэнэ. Иймд $\cos x \neq 0$ тохиолдлыг авч үзье. Тэнцэтгэл бишийн хоёр талыг $\cos^2 x$ -д хуваавал
\begin{equation*}
(\tg x + a)(\tg x + b) \leq \left[1+ \left(\frac{a+b}{2}\right)^2 \right]\sec^2 x
\end{equation*}
болно. $t = \tg x$ гэвэл $\sec^2 x = 1+t^2$ болох ба дээрх тэнцэтгэл бишд орлуулвал
\begin{equation*}
t^2 + (a+b)t + ab \leq \left(\frac{a+b}{2}\right)^2 t^2 + t^2 + \left(\frac{a+b}{2}\right)^2 + 1
\end{equation*}
\begin{equation*}
\left(\frac{a+b}{2}\right)^2 t^2 + 1 - (a+b)t + \left(\frac{a+b}{2}\right)^2 - ab \geq 0
\end{equation*}
\begin{equation*}
\left(\frac{(a+b)t}{2} - 1\right)^2 + \left(\frac{a-b}{2}\right)^2 \geq 0
\end{equation*}
болж батлагдана.

\Problem
\begin{equation*}
|\sin a_1| + |\sin a_2| + \cdots + |\sin a_n| + |\cos (a_1 + a_2 + \cdots + a_n)| \geq 1
\end{equation*}
болохыг батал.

\TheSolution
Математик индукцийн аргыг ашиглан баталъя. $n = 1$ үед
\begin{equation*}
|\sin a_1| + |\cos a_1| \geq \sin^2 a_1 + \cos^2 a_1 = 1
\end{equation*}
учир тэнцэтгэл биш биелнэ.
\begin{equation*}
|\sin a_1| + |\sin a_2| + \cdots + |\sin a_{n+1}| + |\cos (a_1 + a_2 + \cdots + a_{n+1})| \geq 1
\end{equation*}
гэдгийг харуулахын тулд бид
\begin{equation*}
|\sin a_{n+1}| + |\cos (a_1 + a_2 + \cdots + a_{n+1})| \geq |\cos(a_1 + a_2 + \cdots + a_n)|
\end{equation*}
гэж баталъя.

$s_k = a_1 + a_2 + \cdots + a_k$ гэвэл дээрх тэнцэтгэл биш $|\sin a_{n+1}| + |\cos s_{n+1}| \geq |\cos s_n|$ болно. Өнцгүүдийн ялгаварын косинусын томьёог ашиглавал
\begin{align*}
|\cos s_n| &= |\cos(s_{n+1} - a_{n+1})| = \\
&= |\cos s_{n+1}\cos a_{n+1} + \sin s_{n+1}\sin a_{n+1}| = \\
&= |\cos s_{n+1}\cos a_{n+1}| + |\sin s_{n+1}\sin a_{n+1}| \leq \\
&\leq |\cos s_{n+1}| + |\sin a_{n+1}|
\end{align*}
болж батлагдлаа.

\Problem
[Russia 2003, by Nazar Agakhanov]
\begin{align*}
S &= \left\lbrace \sin \alpha, \sin 2\alpha, \sin 3\alpha \right\rbrace \\
T &= \left\lbrace \cos \alpha, \cos 2\alpha, \cos 3\alpha \right\rbrace
\end{align*}
олонлогууд тэнцүү байх бүх боломжит $\alpha$ өнцгийг ол.

\TheSolution
$S, T$  олонлогууд тэнцүү олонлогууд учир элементүүдийн нийлбэр нь хоорондоо тэнцүү байна.
\begin{equation*}
\sin \alpha + \sin 2\alpha + \sin 3\alpha = \cos \alpha + \cos 2\alpha + \cos 3\alpha
\end{equation*}
тэнцүүгийн хоёр талын нийлбэр тус бүрийн нэг болон гуравдугаар нэмэгдэхүүнүүдэд синусуудын болон косинусуудын нийлбэрийн томьёог ашиглавал
\begin{equation*}
2\sin 2\alpha\cos \alpha + \sin 2\alpha = 2\cos 2\alpha \cos \alpha + \cos 2\alpha
\end{equation*}
\begin{equation*}
\sin 2\alpha(2\cos \alpha + 1) = \cos 2\alpha (2\cos \alpha + 1)
\end{equation*}
болно. Хэрэв $2\cos \alpha + 1 = 0$ бол $\cos \alpha  =  -\frac{1}{2}$ болох ба $\alpha = \pm \frac{2}{3}\pi + 2\pi k, k\in \mathbb{Z}$ болно. Гэвч $S$ болон $T$ олонлогууд тэнцүү биш олонлогууд болох тул $2\cos \alpha + 1 \neq 0$ байна. Энэ үед $\sin 2\alpha = \cos 2\alpha \Rightarrow \tg 2\alpha = 1$ болно. Эндээс $\alpha = \frac{\pi}{8} + \frac{\pi k}{2}, k\in \mathbb{Z}$ болно.

\Problem

\Problem
[Canada 1998] $ABC$ гурвалжны $\angle BAC=40^\circ$ ба $\angle ABC = 60^\circ$ болно. $D$, $E$ нь $AC, AB$ дээр  $\angle CBD=40^\circ$, $\angle BCE=70^\circ$ байхаар оршино. $BD$ ба $CE$ нь $F$ цэгт огтлолцох бол $AF\perp BC$ болохыг батал.

\TheSolution
Бодлогын нөхцлөөр $\angle ABD = 20^\circ, \angle BCA = 80^\circ, \angle ACE = 10^\circ$ болно. $A$ цэгээс $BC$ талруу татсан өндрийг $AG$ гэе. Тэгвэл $\angle BAG = 90^\circ - \angle BCA = 10^\circ$ болно. Иймд
\begin{equation*}
\frac{\sin \angle BAG \sin \angle ACE \sin \angle CBD}{\sin \angle CAG \sin \angle BCE \sin \angle ABD} = 
\frac{\sin 30^\circ \sin 10^\circ \sin 40^\circ}{\sin 10^\circ \sin 70^\circ \sin 20^\circ}
= \frac{\frac{1}{2}(\sin 10^\circ)(2\sin 20^\circ \cos 20^\circ)}{\sin 10^\circ \cos 20^\circ \sin 20^\circ} = 1
\end{equation*}
болно. Чевийн теоремийн тригонометр хэлбэрээр $AG=BD=CE$ болно. Иймээс $F \in AG$ бөгөөд $AF \perp BC$ болно.


\Problem
[IMO 1991] $S$ нь $\triangle ABC$ -ны дотор орших цэг бол $\angle SAB$, $\angle SBC$, $\angle SCA$ өнцгүүдийн ядаж нэг нь $30^\circ$ -аас ихгүй байна гэж харуул.

\ASolution
$\alpha =\angle PAB = \angle PBC = \angle PCA$ чанарыг хангах $P$ цэгийг \textbf{(Брокерийн цэг)} авч үзье. $S$ цэг $PAB$, $PBC$, $PCA$ гурвалжнуудын дотор орших тул $\triangle PAB$, $\triangle PBC$, $\triangle PCA$ өнцгүүдийн ядаж нэг нь $\alpha$ өнцгөөс ихгүй байна. Иймд бид $\alpha \leq 30^\circ$ буюу $\sin \alpha \leq \frac{1}{2} \Rightarrow \csc^2 \alpha \geq 4$ гэж харуулахад хангалттай. 28 дугаар бодлогоор бид
\begin{equation*}
\csc^2 \alpha = \csc^2 A + \csc^2 B + \csc^2 C
\end{equation*}
болохыг харуулсан. Cauchy-Schwarz - ийн тэнцэтгэл бишийг ашиглавал
\begin{equation*}
\frac{9}{4}\csc^2 \alpha \geq (\sin^2 A + \sin^2 B + \sin^2 C)(\csc^2 A + \csc^2 B + \csc^2 C)\geq 9
\end{equation*}
гэдгээс $\csc^2 \geq 4$ болж батлагдав.


\ASolution
$S$ цэгээс гурвалжны талууд хүртэлх зайг харгалзан $d_a, d_b, d_c$,  $x= \angle SAB, y=\angle SBC, z=\angle SCA$гэж тэмдэглэвэл
\begin{equation*}
d_c = SA\sin x = SB \sin (B-y)
\end{equation*}
\begin{equation*}
d_a = SB\sin y = SC \sin (C-z)
\end{equation*}
\begin{equation*}
d_b = SC\sin z = SA \sin (A-x)
\end{equation*}
болно. Эдгээрийг хооронд нь үржүүлвэл
\begin{equation}\label{eq:equation1}
\sin x\sin y\sin z = \sin (A-x)\sin(B-y)\sin(C-z)
\end{equation}
болно. Хэрвээ $x+y+z \leq \frac{\pi}{2}$ бол уг бодлого илэрхий. Иймд $x+y+z > \frac{\pi}{2}$ үед $(A-x) + (B-y) + (C-z) < \frac{\pi}{2}$) байна. $f(x) = \ln (\sin x) \vspace{0.5 cm} 0<x<\frac{\pi}{2}$ функцийн хувьд хоёрдугаар зэргийн уламжлал нь $f''(x) = -\csc^2 x$ учир хотгор функц юм. \textbf{Jensen's inequality}-аар
\begin{equation*}
\frac{1}{3}\left(\ln \sin (A-x) + \ln \sin (B-y) + \ln \sin(C-z)\right) \leq \ln \sin \frac{(A-x)+(B-y)+(C-z)}{3}
\end{equation*}
гэдгээс
\begin{equation*}
\ln(\sin(A-x)\sin(B-y)\sin(C-z))^\frac{1}{3} \leq \ln\sin \frac{6}{\pi} = \ln \frac{1}{2}
\end{equation*}
буюу
\begin{equation*}
\sin(A-x)\sin(B-y)\sin(C-z) \leq \frac{1}{8} \Rightarrow \sin x\sin y\sin z \leq \frac{1}{8}
\end{equation*}
болно. Иймд $\sin x, \sin y, \sin z$ -н ядаж нэг нь $\frac{1}{2}$-ээс бага байна.

\ASolution
Өмнөх бодолтын \textbf{Jensen's inequality}-г ашигласанаас хойш үргэлжлүүлбэл (\ref{eq:equation1}) тэнцэтгэлийг
\begin{equation*}
(\sin x \sin y\sin z)^2 = \sin x\sin (A-x)\sin y \sin(B-y)\sin z\sin(C-z)
\end{equation*}
хэлбэрээр бичиж болно. Синусуудын үржвэрийг нийлбэрт хувиргах болон давхар өнцгийн томъёогоор
\begin{equation*}
2\sin x \sin(A-x) = \cos (A-2x) - \cos A \leq 1 -\cos A = 2\sin^2 \frac{A}{2} \Rightarrow \sin x\sin(A-x) \leq \sin^2\frac{A}{2}
\end{equation*}
болно. Иймд 23 дугаар бодлогын (a) -аар 
\begin{equation*}
\sin x \sin y\sin z \leq \sin \frac{A}{2} \sin \frac{B}{2} \sin \frac{C}{2} \leq \frac{1}{8}
\end{equation*}
болж батлагдав.

\Problem
$a=\frac{\pi}{7}$ бол
\begin{enumerate}[(a)]
\item
$\sin^2 3a - \sin^2 a = \sin 2a \sin 3a$ гэж харуул.
\item
$\csc a = \csc 2a + \csc 4a$  гэж харуул.
\item
$\cos a - \cos 2a + \cos 3a$ тооцоол
\item
$\cos a$ нь $8x^3 + 4x^2 -4x-1=0$ тэгшитгэлийн язгуур болохыг батал.
\item
$\cos a$-г иррационал гэж батал.
\item
$\tg a\tg 2a \tg 3a$ тооцоол.
\item
$\tg^2 a + \tg^2 2a + \tg^2 3a$ тооцоол.
\item
$\tg^2 a\tg^2 2a + \tg^2 2a\tg^2 3a + \tg^2 3a\tg^2 a$ тооцоол.
\item
$\ctg^2 a + \ctg^2 2a + \ctg^2 3a$ тооцоол.

\end{enumerate}

\TheSolution
\begin{enumerate}[(a)]
\item
Синусуудын нийлбэр, ялгаварыг үржвэр хэлбэрт оруулан $a=\frac{\pi}{7}$ тохиолдолд $\sin 4a = \sin 3a$ болохыг ашиглавал
\begin{align*}
\sin^2 3a - \sin^2a = (\sin 3a + \sin a)(\sin 3a - \sin a) = \\
= (2\sin 2a\cos a)(2\sin a\cos 2a) = (2\sin 2a\cos 2a)(2\sin a \cos a) = \\
= \sin 4a \sin 2a = \sin 2a \sin 3a
\end{align*}
болж батлагдав.

\item
\begin{equation*}
\frac{1}{\sin a} = \frac{1}{\sin 2a} + \frac{1}{\sin 4a}
\end{equation*}
\begin{equation*}
\frac{1}{\sin a} = \frac{\sin 2a + \sin 4a}{\sin 2a\sin 4a}
\end{equation*}
\begin{equation*}
\sin 2a\sin 4a = \sin a(\sin 2a + \sin 4a)
\end{equation*}
синусуудын нийлбэрийг үржвэр хэлбэрт оруулвал
\begin{equation*}
2\sin a \cos a \sin 4a = \sin a (2\sin 3a\cos a)
\end{equation*}
$\sin 4a = \sin 3a$ тул
\begin{equation*}
2\sin a \cos a \sin 4a = 2\sin a \cos a \sin 4a
\end{equation*}
батлагдлаа.

\item
Хариу нь $\frac{1}{2}$. Иймд $\cos 2a + \cos 4a + \cos 6a = -\frac{1}{2}$ гэж батлахад хангалттай.
Энэ нь
\begin{equation*}
t = \cos 2x + \cos 4x + \cdots + \cos 2nx = -\frac{1}{2}, \vspace{1cm} x=\frac{\pi}{2n+1}
\end{equation*}
тэнцэтгэлийн $n=3$ үеийн тухайн тохиолдол юм. $2\sin x\cos kx = \sin (k+1)x - \sin (k-1)x$ болохыг ашиглавал
\begin{align*}
2t\sin x = 2\sin x (\cos 2x + \cos 4x + \cdots + \cos 2nx) = \\
= [\sin 3x - \sin x]+ [\sin 5x - \sin 3x] + \cdots + [\sin (2n+1) - \sin (2n-1)] =\\
= \sin (2n + 1)x - \sin x = -\sin x
\end{align*}
болж батлагдлаа.

\item
$3a + 4a = \pi \Rightarrow \sin 3a = \sin 4a$  болно. 
\begin{align*}
\sin a(3-4\sin^2 a) = 2\sin 2a \cos 2a = 4\sin a \cos a \cos 2a\\
3 - 4(1-\cos^2 a) = 4\cos a (2\cos^2 a - 1)
\end{align*}
Иимээс
\begin{equation}
8\cos^3 a - 4\cos^2 a - 4\cos a + 1 = 0
\end{equation}
болно. $u=2\cos a$ -г куб тэгшитгэлийн язгуур гэвэл
\begin{equation*}
u^3-u^2-2u + 1=0
\end{equation*}
болно. Гауссын леммээр дээрх куб тэгшитгэлийн $-1, 1$ гэсэн рационал шийдүүдтэй байх боломжтой. Гэвч эдгээр нь язгуур биш учир энэ тэгшитгэлд рационал язгуур байхгүй. Иймд $\cos a$ иррационал тоо учир (e) батлагдлаа.

\stepcounter{enumi}

\item
$3a + 4a = 0 \Rightarrow \tg 3a + \tg 4a = 0$ болно. Өнцгүүдийн нийлбэрийн тангенсийн томъёогоор
\begin{equation*}
\frac{\tg a + \tg 2a}{1-\tg a\tg 2a} + \frac{2\tg 2a}{1-\tg^2 2a} = 0
\end{equation*}
\begin{equation*}
\tg a + 3\tg 2a -3\tg a\tg^2 2a - \tg^3 2a = 0
\end{equation*}
$\tg a = x$ гэвэл $\tg 2a = \frac{2\tg a}{1-\tg^2 a} = \frac{2x}{1-x^2}$ болох ба
\begin{equation*}
x + \frac{6x}{1-x^2} - \frac{12 x^3}{(1-x^2)^2} - \frac{8x^3}{(1-x^2)^3} = 0
\end{equation*}
\begin{equation*}
(1-x^2)^3 + 6(1-x^2)^2 - 12x^2(1-x^2) - 8x^2 = 0
\end{equation*}
\begin{equation*}
x^6 - 21x^4 + 35x^2 - 7 = 0
\end{equation*}
болно.

\end{enumerate}



\end{document}
